\newpage
\section{第八次课后作业}
\begin{tcolorbox}[breakable,colback=blue!5!white,colframe=blue!75!black,
 title= 填空题]

码 $ C $ 至多可纠正 $ t $ 个错误的充分必要条件是码 $ C $ 的最小距离 $ d(C)= $ $\underline{\hspace{2em}}$.

 \tcblower
定理:码$C$至多可纠正$t$个错误的充分必要条件为$d(C)=2t+1$或$d(C)=2t+2$.
 \end{tcolorbox}



 \begin{tcolorbox}[breakable,colback=blue!5!white,colframe=blue!75!black,
 title= 填空题]

对于任意二元 $ (3, M, 2) $ 码, 一定有 $ M \leq $ $\underline{\hspace{2em}}$.

 \tcblower
使用Singleton界定理可以帮助我们确定编码理论中线性码的一个上界.Singleton界定理指出,对于任意的 \(q\)-元 \((n, M, d)\) 码,其码长为 \(n\)、码字数为 \(M\) 以及最小汉明距离为 \(d\),我们有:

\[
M \leq q^{n-d+1}
\]

对于一个二元 \((3, M, 2)\) 码,其中 \(q = 2\)(因为是二元码)、\(n = 3\)(码字长度为3)和 \(d = 2\)(最小汉明距离为2),因此
   \[
   M \leq 2^{3-2+1} = 2^2 = 4
   \]
 \end{tcolorbox}



 \begin{tcolorbox}[breakable,colback=blue!5!white,colframe=blue!75!black,
 title= 填空题]

设二元 $ [4,2] $ 线性码 $ L=\{0000,1100,0011,1111\} $, 则 $ L $ 的最小距离为 $\underline{\hspace{2em}}$.

 \tcblower
为了找出给定的二元线性码 \( L = \{0000, 1100, 0011, 1111\} \) 的最小距离,我们需要计算码集中任意两个不同码字之间的汉明距离,然后找出这些距离中的最小值.

汉明距离是指两个码字在相应位上不同的位数.对于线性码,最小距离也可以通过找出除了零码字以外的码字的最小重量(即非零位的数量)来得到,因为线性码的任意两个码字的差也是该码中的一个码字.

我们用两种方法分别来计算:

1. 码字重量分析:

   \( 0000 \) 的重量为 \( 0 \)(无需考虑,因为我们寻找的是非零码字的最小重量).
   
    \( 1100 \) 的重量为 \( 2 \).
    
    \( 0011 \) 的重量为 \( 2 \).
    
    \( 1111 \) 的重量为 \( 4 \).

2. 码字间的汉明距离:

    从 \( 0000 \) 到 \( 1100 \),距离为 \( 2 \).
    
    从 \( 0000 \) 到 \( 0011 \),距离为 \( 2 \).
    
    从 \( 0000 \) 到 \( 1111 \),距离为 \( 4 \).
    
    从 \( 1100 \) 到 \( 0011 \),距离为 \( 4 \).
    
    从 \( 1100 \) 到 \( 1111 \),距离为 \( 2 \).
    
    从 \( 0011 \) 到 \( 1111 \),距离为 \( 2 \).

由上面的计算可见,码字之间的最小汉明距离是 \( 2 \),这也是该码的最小距离.因此,给定的二元线性码 \( L \) 的最小距离为 \( 2 \).
 \end{tcolorbox}


 \begin{tcolorbox}[breakable,colback=blue!5!white,colframe=blue!75!black,
 title= 填空题]

对于任意 $ n \geq 1, \quad A_{q}(n, n)= $  $\underline{\hspace{2em}}$.

 \tcblower
设 $ C $ 是一个 $ q $ 元 $ (n, M, n) $ 码, 则 $ \forall x, y \in C, x=\left(x_{1}, \cdots, x_{n}\right), y=\left(y_{1}, \cdots, y_{n}\right), x_{i} \neq y_{i}, i=  1, \cdots, n $, 因此, 所有码字在一个固定分量位置上出现的字符一定互不相同, 于是 $ M \leq q $. 由此可知 $ A_{q}(n, n) \leq q $, 又码长为 $ n $ 的 $ q $ 元重复码是一个 $ q $ 元 $ (n, q, n) $ 码, 故 $ A_{q}(n, n)=q $.
 \end{tcolorbox}


 \begin{tcolorbox}[breakable,colback=blue!5!white,colframe=blue!75!black,
 title= 解答题]

(1) 证明: 对任意三元 $ (3, M, 2) $ 码, 一定有 $ M \leq 9 $.

(2) 证明: 三元 $ (3,9,2) $ 码一定存在. 于是, $ A_{3}(3,2)=3^{2} $.

(3) 证明: $ A_{q}(3,2)=q^{2} $, 其中 $ q \geq 2, q $ 是素数的幂次方.

 \tcblower
(1) 对于任意的三元 $(3, M, 2)$ 码,根据 Singleton界,$A_q(n, d) \leq q^{n-d+1}$, 将$n=3, d=2, q=3$(因为是三元码,所以$q=3$)代入上述公式:
$A_3(3, 2) \leq 3^{3-2+1} = 3^{2} = 9$.这表明在保证任意两个码字之间的汉明距离至少为 $2$ 的情况下,码字总数$M$不能超过 $9$.
因此,任何$(3, M, 2)$码的码字数量$M$最大为 $9$, 这就证明了$M \leq 9$.

 (2) 为了证明,我们需要构造一个具体的 $(3,9,2)$ 码.考虑以下码集$C$:
$$C = \{000, 111, 222, 012, 021, 120, 102, 210, 201\}$$
这 $9$ 个码字确保了任意两个码字之间的汉明距离至少为 $2$. 由于我们找到了一个有效的$(3,9,2)$码,因此$A_3(3,2) \geq 9$.结合前面的Singleton界结果$A_3(3,2) \leq 9$,可以断定$A_3(3,2) = 9$.

 (3) 使用Singleton界:
\[ A_q(3, 2) \leq q^{3-2+1} = q^{2} \]
我们需要构造一个$q$元$(3, q^2, 2)$码来证明存在性.考虑码集:
\[ C = \{(a, b, a+b) \mid a, b \in \mathbb{F}_q\} \]
其中$\mathbb{F}_q$是有$q$个元素的有限域.这种构造中,每个码字形式为$(a, b, a+b)$,其中每个$a$和$b$可以独立选择,因此共有$q^2$个码字.即$|C| = q^2$.由于对于任何两个不同的码字$(a, b, a+b)$和$(a', b', a'+b')$,至少在两个坐标上有不同(如果$a \neq a'$那么$a+b \neq a'+b'$),所以这种码的最小汉明距离$d(C)=2$.

因此,$A_q(3, 2) = q^2$.
 \end{tcolorbox}


  \begin{tcolorbox}[breakable,colback=blue!5!white,colframe=blue!75!black,
 title= 解答题]

试说明对于二元重复码 $ C=\{00 \cdots 0,11 \cdots 1\} $, 它是一个二元 $ (n, 2, n) $ 码, 当 $ n $ 为奇数时, $ C $ 是完备码. 另外, 只含一个码字的码以及由 $ V(n, q) $ 构成的 $ q $ 元 $ \left(n, q^{n}, 1\right) $ 码都是完备码.

 \tcblower
设 $ C $ 是一个q元 $ (n, M, 2 t+1) $ 码. 如果
$$
M\left\{\left(\begin{array}{l}
n \\
0
\end{array}\right)+\left(\begin{array}{l}
n \\
1
\end{array}\right)(q-1)+\left(\begin{array}{l}
n \\
2
\end{array}\right)(q-1)^{2}+\cdots+\left(\begin{array}{l}
n \\
t
\end{array}\right)(q-1)^{t}\right\}=q^{n},
$$
则称 $ C $ 为完备码 (perfect code).

(1)对于码长为 $ n $ 的二元重复码
$$
C_{1}=\{\underbrace{00 \cdots 0}_{n}, \underbrace{11 \cdots 1}_{n}\} .
$$
$$
\begin{aligned}
&2\left\{\left(\begin{array}{l}
n \\
0
\end{array}\right)+\left(\begin{array}{l}
n \\
1
\end{array}\right)+\right.  \left.\left(\begin{array}{l}
n \\
2
\end{array}\right)+\cdots+\left(\begin{array}{l}
n \\
t
\end{array}\right)\right\} \\
= & \left(\begin{array}{l}
n \\
0
\end{array}\right)+\left(\begin{array}{l}
n \\
1
\end{array}\right)+\left(\begin{array}{l}
n \\
2
\end{array}\right)+\cdots+\left(\begin{array}{c}
n \\
l
\end{array}\right)+\left(\begin{array}{c}
n \\
t+1
\end{array}\right)+\cdots \\
& +\left(\begin{array}{c}
n \\
n-2
\end{array}\right)+\left(\begin{array}{c}
n \\
n-1
\end{array}\right)+\left(\begin{array}{l}
n \\
n
\end{array}\right) \\
& =(1+1)^{n} \\
& =2^{n} .
\end{aligned}
$$
因此, 当码长 $ n $ 为奇数时, 二元重复码 $ C_{1} $ 是一个完备的 $ (n, 2, n) $ 码.

(2)对于只含有一个码字的码 $C_{2}=\{x\} \subset V(n, q)$, 当在信道发送端发送码字 $x$ 后, 在信道接收端不管接收到什么向量都将译为码字 $x$. 这就是说, 码 $ C_{2} $ 可以检查和纠正码字在信道传输过程中发生的任何数目的锴误. 因此, 码 $C_{2}$ 可纠正的错误数目为 $ t =n $. 显然,
$$
\begin{aligned}
\left(\begin{array}{l}
n \\
0
\end{array}\right)+\left(\begin{array}{l}
n \\
1
\end{array}\right)(q-1) & +\left(\begin{array}{l}
n \\
2
\end{array}\right)(q-1)^{2}+\cdots+\left(\begin{array}{l}
n \\
n
\end{array}\right)(q-1)^{n} \\
& =(1+(q-1))^{n} \\
& =q^{n} .
\end{aligned}
$$
因此, 只含有一个码字的码 $ C_{2} $ 是完备码.

(3)对于码 $ C_{3}=V(n, q) $, 其码字个数为 $ q^{n} $. 最小距离为 1 , 可纠正的锴误数目为 $ t=0 $. 显然, 对于 $q $ 元 $ \left(n, q^{n}, 1\right) $ 码$ C_{3} $. 满足定义. 因此, $ C_3=V(n, q) $ 是完备码.

 \end{tcolorbox}



   \begin{tcolorbox}[breakable,colback=blue!5!white,colframe=blue!75!black,
 title= 解答题]

对于任意 $ n \geq 1 $, 试确定 $ A_{q}(n, n) $.

 \tcblower


设 $ C $ 是一个 $ q $ 元 $ (n, M, n) $ 码,则 $ C $ 中任意两个不同的码字 $ \boldsymbol{x} $ 和 $ \boldsymbol{y} $ 的 Hamming 距离都是 $ n $, 也就是说, $ \boldsymbol{x} $ 和 $ \boldsymbol{y} $ 的 $ n $ 个分量一定互不相同. 于是,对于任意一个分量位置 $ i$, $C $ 中 $ M $ 个码字的第 $ i $ 个分量一定互不相同. 因此, $ M \leq q $. 另一力面, 我们已经知道码长为 $ n $ 的 $ q $ 元重复码是一个 $ (n, q, n) $ 码. 因此, $ A_{q}(n, n)=q $.
 \end{tcolorbox}



\newpage
 \begin{tcolorbox}[breakable,colback=blue!5!white,colframe=blue!75!black,
 title= 解答题]

试说明对于二元重复码 $ C=\{00 \cdots 0,11 \cdots 1\} $, 它是一个二元 $ (n, 2, n) $ 码, 当 $ n $ 为奇数时, $ C $ 是完备码. 另外, 只含一个码字的码以及由 $ V(n, q) $ 构成的 $ q $ 元 $ \left(n, q^{n}, 1\right) $ 码都是完备码.

 \tcblower
(1) 二元重复码 \( C=\{00 \cdots 0, 11 \cdots 1\} \) 是一个二元 \( (n, 2, n) \) 码,当 \( n \) 为奇数时, \( C \) 是完备码.

二元重复码 \( C \) 包含两个码字:全0和全1,每个码字长度为 \( n \).
最小汉明距离\( d = n \) 是因为两个码字在每个位上都不同.

验证完备码条件:
完备码的定义要求:
$$
M \left( \sum_{i=0}^t \binom{n}{i} \right) = 2^n,
$$
其中 \( M = 2 \) 是码字数量,\( t = \frac{n-1}{2} \).

由于二项式定理给出:
$$
\sum_{i=0}^n \binom{n}{i} = 2^n,
$$
利用对称性,当 \( n \) 为奇数时:
$$
\sum_{i=0}^{\frac{n-1}{2}} \binom{n}{i} = \sum_{i=\frac{n+1}{2}}^n \binom{n}{i},
$$
因此,
$$
2 \left( \sum_{i=0}^{\frac{n-1}{2}} \binom{n}{i} \right) = 2^n,
$$
满足完备码的条件.因此,当 \( n \) 为奇数时,二元重复码 \( C \) 是完备码.


(2) 只含一个码字的码 \( C_2=\{x\} \) 以及由 \( V(n, q) \) 构成的 \( q \) 元 \( (n, q^n, 1) \) 码都是完备码.

只含一个码字的码 \( C_2 \): 任何接收的向量都被解释为唯一的码字 \( x \). 纠正的错误数目 \( t = n \)(最大可能的错误数目).

完备码条件为:
$$
\left( \sum_{i=0}^n \binom{n}{i} (q-1)^i \right) = q^n.
$$
由于二项式定理,上式变为 \( (1 + (q-1))^n = q^n \),显然满足.因此 \( C_2 \) 是完备码.

(3) 由 \( V(n, q) \) 构成的 \( q \) 元 \( (n, q^n, 1) \) 码 \( C_3 \): 包括所有 \( n \)-维向量,错误纠正个数 \( t = 0 \).
 由于覆盖了整个 \( n \)-维空间,满足完备码条件,即有 \( q^n = q^n \),所以 \( C_3 \) 也是完备码.
\end{tcolorbox}