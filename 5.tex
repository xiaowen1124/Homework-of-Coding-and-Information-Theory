\section{第五次课后作业}

\begin{tcolorbox}[breakable,colback=blue!5!white,colframe=blue!75!black,
 title= 解答题]
 对下面给定的概率分布和基数, 找出一个Huffman编码, 并求平均码长.
$$
p=\{0.1,0.1, \cdots, 0.1\}, r=3 .
$$

\tcblower
解: 首先确定 $ {k} $ 的值,$a=10,r=3$.
$$
k=\operatorname{I n t}_{+}\left(\frac{a-1}{r-1}\right)=\operatorname{I n t}_{+}\frac{9}{2}=5.
$$
再确定第一列最后几个分量相加:$a-(k-1) r+k-1 =10-(5-1) \times 3+5-1=2 $.从第三列开始每次将最后 $r$ 个分量相加,并按大小排序放入下一奇数列.于是我们构造Huffman编码为

\begin{center}
\begin{tabular}{ll||ll||ll||ll||ll} 
\hline
概率 & 码 & 概率 & 码 & 概率 & 码 &概率 & 码 &概率 & 码 \\
\hline
0.1 & 01 & $\boxed{0.2}$ & 00 & $\boxed{0.3}$ & 2 &$\boxed{0.3}$ & 1 &\boxed{0.4} &0  \\
0.1 & 02 & 0.1 & 01 &  0.2  & 00 &0.3 & 2 &0.3 &1 \\
0.1 & 10 & 0.1 & 02 & 0.1 & 01 &0.2 & 00 &0.3 &2 \\
0.1 & 11 & 0.1 & 10 & 0.1 & 02 &0.1 &01  & & \\
0.1 & 12 & 0.1 & 11 & 0.1 & 10 &0.1 &02  & & \\
0.1 & 20 & 0.1 & 12 & 0.1 & 12 & &  & &  \\
0.1 & 21 & 0.1 & 20 & 0.1 &  &  & &\\
0.1 & 22 & 0.1 & 21 &  &  & & & &\\
0.1 & 000 & 0.1 & 22 &  &  & & & &\\
0.1 & 001 &  &  & & & & & &\\
\hline
\end{tabular}
\end{center}

平均码长:
$$L(\mathscr{S}, f)=0.8 \times 2+0.2 \times 3=2.2 $$


\end{tcolorbox}


\begin{tcolorbox}[breakable,colback=blue!5!white,colframe=blue!75!black,
 title= 解答题]
 对下面给定的概率分布和基数,找出一个Huffman编码, 并求平均码长.
$$
\begin{array}{l}
p=\{0.3,0.1,0.1,0.1,0.1,0.06,0.05,0.05,0.05,0.04,0.03,0.02\}, 
\quad r=4 .
\end{array}
$$

\tcblower
解: 首先确定 $ {k} $ 的值,$a=12,r=4$.
$$
k=\operatorname{I n t}_{+}\left(\frac{a-1}{r-1}\right)=\operatorname{I n t}_{+}\frac{11}{3}=4.
$$
再确定第一列最后几个分量相加:$a-(k-1) r+k-1 =12-(4-1) \times 4+4-1=3 $.从第三列开始每次将最后 $r=4$ 个分量相加,并按大小排序放入下一奇数列,用方框标出.于是我们构造Huffman编码为

\begin{center}
\begin{tabular}{ll||ll||ll||ll} 
\hline
概率 & 码 & 概率 & 码 & 概率 & 码& 概率 & 码 \\
\hline
0.3 & 1 & 0.3 & 1 & 0.3 & 1&$\boxed{0.39}$ &0 \\
0.1 & 3 & 0.1 & 3 & $\boxed{0.21}$ & 2 &0.3 &1 \\
0.1 & 00 & 0.1 & 00 & 0.1 & 3 &0.21 &2 \\
0.1 & 01 & 0.1 & 01 &0.1 &00 &0.1 &3 \\
0.1 & 02 & 0.1 & 02 &0.1 &01 & & \\
0.06 & 20 &$\boxed{0.09}$ &03 &0.1 &02 & & \\
0.05 & 21 &0.06 &20 &0.09 &03 & & \\
0.05 & 22 &0.05 &21 & & & & \\
0.05 & 23 &0.05 &22 & & & & \\
0.04 & 030 &0.05 &23 & & & & \\
0.03 & 031 & & & & & & \\
0.02 & 032 & & & & & & \\
\hline
\end{tabular}
\end{center}
平均码长:
$$ L(\mathscr{S}, f)=0.4\times 1+0.51 \times 2+0.09 \times 3=1.69 $$
\end{tcolorbox}


\newpage
\begin{tcolorbox}[breakable,colback=blue!5!white,colframe=blue!75!black,
 title= 解答题]
 判断是否存在即时码具有以下的基数和码字长度, 如果有,试构造出一个这样的码.
$ r=3 $, 长度 $ 1,1,2,4,4,5 $.

\tcblower
我们计算Kraft和:
$$
\sum_{i=1}^{6} 3^{-\ell_{i}}=2 \times 3^{-1}+1 \times 3^{-2}+2 \times 3^{-4}+1 \times 3^{-5}=\frac{2}{3}+\frac{1}{9}+\frac{2}{81}+\frac{1}{243}=\frac{196}{243}<1
$$
满足Kraft不等式,因此存在即时码. 下面构造一个即时码:
$$
\begin{array}{llllll}
u_{1,1}=0 & u_{1,2}=1 & &&&0,1\\
u_{2,1,1}=2 & u_{2,1,2}=0 &&&& (2,0)  \\
u_{4,1,1}=2 & u_{4,1,2}=1 & u_{4,1,3}=0&u_{4,1,4}=0&&(2,1,0,0)  \\
u_{4,2,1}=2 & u_{4,2,2}=1 & u_{4,2,3}=0&u_{4,2,4}=1&&(2,1,0,1)  \\
u_{5,1,1}=2 & u_{5,1,2}=1 & u_{5,1,3}=1 & u_{5,1,4}=0& u_{5,1,5}=0& (2,1,1,0,0) \\
\end{array}
$$
故此即时码为
$$
\{0,1,20,21,2100,2101,21100\}
$$
\end{tcolorbox}


\begin{tcolorbox}[breakable,colback=blue!5!white,colframe=blue!75!black,
 title= 解答题]
对下面给定的概率分布和基数,找出一个Huffman编码,并求平均码长.
$$
p=\{0.3,0.2,0.2,0.1,0.1,0.1\}, \quad r=2 .
$$

\tcblower
解: 首先确定 $ {k} $ 的值,$a=6,r=2$.
$$
k=\operatorname{I n t}_{+}\left(\frac{a-1}{r-1}\right)=\operatorname{I n t}_{+}\frac{5}{1}=5.
$$
再确定第一列最后几个分量相加:$a-(k-1) r+k-1 =6-(5-1) \times 2+5-1=2 $.从第三列开始每次将最后 $r=2$ 个分量相加,并按大小排序放入下一奇数列,用方框标出.于是我们构造Huffman编码为

\begin{center}
    \begin{tabular}{ll||ll||ll||ll||ll}
\hline 概率 & 码 & 概率 & 码 & 概率 & 码 & 概率 & 码& 概率 & 码\\
\hline 0.3 & 01 & 0.3 &01 & $\boxed{0.3}$ & 00 &$\boxed{0.4}$ &1 &$\boxed{0.6}$ &0\\
0.2 & 11 & $\boxed{0.2}$ & 10 & 0.3 & 01 &0.3 &00 &0.4 &1  \\
0.2 & 000 & 0.2 & 11 & 0.2 & 10 &0.3 &01 & &  \\
0.1 & 001 & 0.2 & 000 &0.2 &11 & & & &  \\
0.1 & 100 & 0.1 & 001 & & & & & &  \\
0.1 & 101 & & & & & & & &  \\
\hline
\end{tabular}
\end{center}
平均码长:
$$ L(\mathscr{S}, f)=0.5\times 2+0.5 \times 3=2.5 $$
\end{tcolorbox}

 