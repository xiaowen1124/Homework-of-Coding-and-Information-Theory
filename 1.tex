
\section{第一次课后作业}

\begin{tcolorbox}[breakable,colback=blue!5!white,colframe=blue!75!black,
 title= 单选题]
  概率分布 $ \bar{p}=\left\{p_{1}, p_{2}, \cdots, p_{a}\right\} $ 是一个确定性分布 为熵 $ H\left(p_{1}, p_{2}, \cdots, p_{a}\right)=0 $ 的(  ) 条件.

    
(A) 充分条件;
(B) 必要条件;
(C) 充分必要条件;
(D) 既不充分也不必要.
 \tcblower

概率分布 $\bar{p}=\left\{p_{1}, p_{2}, \cdots, p_{a}\right\}$ 是一个确定性分布,即所有的概率都为1或0,因此熵 $H\left(p_{1}, p_{2}, \cdots, p_{a}\right)=0$.

反之,若$H\left(p_{1}, p_{2}, \cdots, p_{a}\right)=0 \text {, 则由 } H\left(p_{1}, p_{2}, \cdots, p_{a}\right) \text { 的 }
$
定义可知, $ \forall i, p_{i} \log _{c} p_{i}=0 $, 或者 $ p_{i}=0 $, 或者 $ \log _{c} p_{i}=0 $, 由于 $ \sum\limits_{i=1}^{a} p_{i}=1, p_{i} \geq 0 $, 存在 $ i $ 使得 $ p_{i}=1 $, 而其它 $ p_{j}=0 $, 因此 $ \bar{p} $ 必为确定型分布.

所以,答案是:(C) 充分必要条件


\end{tcolorbox}


\begin{tcolorbox}[breakable,colback=blue!5!white,colframe=blue!75!black,
 title= 单选题]
 设 $ \xi $ 是一个二元随机变量, 即 $ \mathscr{X}=\{0,1\} $,令 $ p(\xi=1)=p, p(\xi=0)=1-p $. 则有二元熵函数 $ H(p)=p \log _{2} \frac{1}{p}+(1-p) \log _{2} \frac{1}{1-p} $,则当 $ p=$(  ) 时, $ H(p) $ 达到最大值.

(A) $ 0  $ (B) $ \frac{1}{4} $;
(C) $ \frac{1}{2} $;(D) 1 .
  \tcblower
  首先,计算 $ H(p) $ 的导数:
$$
\begin{aligned}
H(p) & =-p \log_2 p-(1-p) \log_2 (1-p) \\
H^{\prime}(p) & =-\log_2 p-p \cdot \frac{1}{\ln 2 \cdot p}+\log_2 (1-p)+\frac{1}{1-p} \cdot \frac{1-p}{\ln 2} \\
& =-\log_2 p-\frac{1}{\ln 2}+\log_2 (1-p)+\frac{1}{\ln 2} \\
& =-\log_2 p+\log_2 (1-p) \\
& =\log_2 \frac{1-p}{p}
\end{aligned}
$$
令导数等于零,解方程 $ \log _{2} \frac{1-p}{p}=0 $,得到 $ p=\frac{1}{2} $.

接下来,我们来验证 $ p=\frac{1}{2} $ 是 $ H(p) $ 的最大值点还是最小值点.我们可以通过二阶导数的符号来判断.计算二阶导数:
$$
\begin{aligned}
\frac{d^{2} H(p)}{d p^{2}} &=\frac{d}{d p}\left[\log _{2} \frac{1-p}{p}\right] \\
&=\frac{1}{p(p-1)\ln 2}
\end{aligned}
$$
当 $ p=\frac{1}{2} $ 时,$ \frac{1}{p(p-1)\ln 2}<0 $,所以 $ p=\frac{1}{2} $ 是 $ H(p) $ 的最大值点.

因此,当 $ p=\frac{1}{2} $ 时,$ H(p) $ 达到最大值.选项 (C) $ \frac{1}{2} $ 是正确答案.
\end{tcolorbox}

\begin{tcolorbox}[breakable,colback=blue!5!white,colframe=blue!75!black,
 title= 单选题]
若 $ H(\xi, \eta)=H(\xi)+H(\eta) $ ,则随机变量 $ \xi $与 $\eta$ 的关系(  ).

A. $ \xi $ 由 $ \eta $ 决定;

B. $ \eta $ 由$ \xi $决定;

C. $ \xi $ 与 $ \eta $ 相互独立.

  \tcblower

当两个随机变量 $\xi$ 和 $\eta$ 相互独立时,它们的联合概率分布可以表示为它们各自的边缘概率分布的乘积,即 $p(\xi, \eta) = p(\xi) \cdot p(\eta)$.根据熵的定义,随机变量的熵可以表示为 $H(\xi) = -\sum\limits_{{X}} p(x) \log p(x)$,其中 $x$ 是随机变量 $\xi$ 的取值.同样地,$H(\eta) = -\sum\limits_{{Y}} p(y) \log p(y)$,其中 $y$ 是随机变量 $\eta$ 的取值.当两个随机变量相互独立时,它们的联合熵可以表示为 $H(\xi, \eta) = -\sum\limits_{X}\sum\limits_{Y} p(x, y) \log p(x, y)$.由于它们相互独立,联合概率分布可以拆分为各自的边缘概率分布的乘积,即 $p(x, y) = p(x) \cdot p(y)$.代入联合熵的定义中,我们有:

$$
\begin{aligned}
H(\xi, \eta) &= -\sum_{X}\sum_{Y} p(x, y) \log p(x, y) \\
&= -\sum_{{X}}\sum_{Y} p(x) \cdot p(y) \log (p(x) \cdot p(y)) \\
&= -\sum_{X}\sum_{Y} p(x) \cdot p(y) (\log p(x) + \log p(y)) \\
&= -\sum_{X}\sum_{Y} p(x) \cdot p(y) \log p(x) - \sum_{X}\sum_{Y} p(x) \cdot p(y) \log p(y) \\
&= -\sum_{X} p(x) \log p(x) - \sum_{Y} p(y) \log p(y) \\
&= H(\xi) + H(\eta)
\end{aligned}
$$

因此,当 $H(\xi, \eta) = H(\xi) + H(\eta)$ 时,可以得出 $\xi$ 和 $\eta$ 是相互独立的.

\end{tcolorbox}



\begin{tcolorbox}[breakable,colback=blue!5!white,colframe=blue!75!black,
 title= 单选题]
$  \mathrm{H}(\xi, \eta)=\mathrm{H}(\xi) $ ,则随机变量 $ \xi $ 与 $ \eta $ 的关系 ( ).

A. $ \xi $ 由 $ \eta $ 决定;

B. $ \xi $ 与相互独立;

C. $ \eta $ 由$ \xi $决定.
  \tcblower

根据题目中的信息 $ \mathrm{H}(\xi, \eta)=\mathrm{H}(\xi) $,这意味着给定 $ \xi $ 的情况下, $ \eta $ 的条件熵为零,即 $ \mathrm{H}(\eta|\xi) = 0 $.这表明在已知 $ \xi $ 的情况下, $ \eta $ 是确定的,因此可以得出结论: $ \eta $ 由 $ \xi $ 决定.因此,答案是 C. $ \eta $ 由 $ \xi $ 决定.

\end{tcolorbox}


\begin{tcolorbox}[breakable,colback=blue!5!white,colframe=blue!75!black,
 title=计算题]
计算 $ H\left(\frac{1}{a}, \frac{1}{a}, \cdots, \frac{1}{a}, \frac{2}{a}, \frac{2}{a}\right) $


 \tcblower
由 $ \Sigma P_{i}=1 $ 知,含 $ (a-4) $ 个 $ \frac{1}{a}$ , 2 个$ \frac{2}{a} $ ,总共$ (a-2) $ 项, 于是
$$
\begin{aligned}
H\left(\frac{1}{a}, \frac{1}{a}, \cdots, \frac{1}{a}, \frac{2}{a}, \frac{2}{a}\right) & =\sum_{i=1}^{a-2} p_{i} \cdot \log \frac{1}{p_{i}} \\
& =\sum_{i=1}^{a-4} \frac{1}{a} \log a+2 \cdot \frac{2}{a} \log \frac{a}{2} \\
& =\frac{a-4}{a} \cdot \log a+\frac{4}{a} \log \frac{a}{2} \\
& =\frac{a-4}{a} \cdot \log a+\frac{4}{a} \log a-\frac{4}{a} \log 2 \\
& =\log a-\frac{4}{a} \log 2
\end{aligned}
$$
\end{tcolorbox}


\begin{tcolorbox}[breakable,colback=blue!5!white,colframe=blue!75!black,
 title=计算题]
设两只口袋中各有20个球,第一支口袋中有10个白球,5个黑球和5 个红球; 第二只口袋中有8个白球, 8 个黑球和 4 个红球,从每只口袋中各取一个球, 试判断哪一个结果的不肯定性更大.


 \tcblower
当我们要判断哪个结果的不确定性更大时,可以使用熵来衡量.
首先,我们将第一只口袋的球的颜色作为随机变量 $\xi_1$,它的概率分布为:
$$
\xi_1 \sim \left(\begin{array}{ccc}\text{白} & \text{黑} & \text{红} \\ \frac{1}{2} & \frac{1}{4} & \frac{1}{4}\end{array}\right)
$$
其中,$\frac{1}{2}$ 表示白球的概率,$\frac{1}{4}$ 表示黑球的概率,$\frac{1}{4}$ 表示红球的概率.

计算第一只口袋的熵 $H(\xi_1)$:
$$ H\left(\xi_{1}\right)=\frac{1}{2} \log 2+2 \times \frac{1}{4} \log 4=\frac{1}{2}+1=1.5  \text{ bits}
$$
接下来,我们将第二只口袋的球的颜色作为随机变量 $\xi_2$,它的概率分布为:
$$
\xi_2 \sim \left(\begin{array}{ccc}\text{白} & \text{黑} & \text{红} \\ \frac{2}{5} & \frac{2}{5} & \frac{1}{5}\end{array}\right)
$$
其中,$\frac{2}{5}$ 表示白球的概率,$\frac{2}{5}$ 表示黑球的概率,$\frac{1}{5}$ 表示红球的概率.

计算第二只口袋的熵 $H(\xi_2)$:

$$
\begin{aligned}
H\left(\xi_{2}\right) & =\frac{4}{5} \log \frac{5}{2}+\frac{1}{5} \log 5 \\
& =\frac{4}{5}(\log 5-\log 2)+\frac{1}{5} \log 5 \\
& =\frac{4}{5} \log 5+\frac{1}{5} \log 5-\frac{4}{5} \\
& =\log 5-\frac{4}{5} \approx 2.32-0.8 \\
& =1.52 \text { bits }
\end{aligned}
$$
比较 $H(\xi_1)$ 和 $H(\xi_2)$ 的值,我们可以得出结论:第二只口袋的结果的不确定性更大,因为它的熵值更大. 
\end{tcolorbox}



