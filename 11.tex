\newpage
 \begin{tcolorbox}[breakable,colback=blue!5!white,colframe=blue!75!black,
 title= 解答题]
设三元线性码 $ L $ 的生成矩阵为
$
G=\left(\begin{array}{llll}
1 & 0 & 1 & 1 \\
0 & 1 & 1 & 2
\end{array}\right)
$.试求 $ L $ 的最小距离, 并证明 $ L $ 是完备码.

\tcblower
 $ G=\left(\begin{array}{ll|ll}1 & 0 & 1 & 1 \\ 0 & 1 & 1 & 2\end{array}\right)=\left(I_{2} \mid A\right) $, 故 $ L $ 的校验阵 $H= \left(-A^{T} \mid I_{2}\right)=\left(\begin{array}{llll}2 & 2 & 1 & 0 \\ 2 & 1 & 0 & 1\end{array}\right)$. 

$ H $ 中任意两列线性无关, 存在第 1,2 ,4 列线性相关,根据定理知,$d(L)=3$. 

$L$ 为一个三元$(4,4,3)$ 码,由于
$$
3^{2}\left(\binom{4}{0}+\binom{4}{1}(3-1)\right)=3^{4}
$$
因此, $ L $ 是完备码.
\end{tcolorbox}


 \begin{tcolorbox}[breakable,colback=blue!5!white,colframe=blue!75!black,
 title= 解答题]

设二元线性码 $ L $ 的生成矩阵为
$
G=\left(\begin{array}{lllll}
1 & 1 & 0 & 1 & 0 \\
0 & 1 & 0 & 1 & 0
\end{array}\right) .
$
试求 $ L $ 的标准阵, 并对信道接收端接收到的字11111和10000分别进行译码.
\tcblower

易知 $L$  为一个 2 元$[5,2]$线性码, $|L|=q^{k}=2^{2}=4$ .
$L=x G=\left(x_{1}, x_{2}\right)\left(\begin{array}{lllll}
1 & 1 & 0 & 1 & 0 \\
0 & 1 & 0 & 1 & 0
\end{array}\right)$,
$(x_{1}, x_{2})$ 分别取$(0,0),(0,1),(1,0),(1,1)$ . 计算可得 $L=\{00000,01010,11010,10000\}$ , 于是标准阵: 

$$
\begin{array}{lllll}
00000 & 01010 & 11010 & \underline{10000}  &\\
01000 & 00010 & 10010 & 11000 & a_{1}+L  \\
00100 & 01110 & 11110 & 10100 & a_{2}+L \\
00001 & 01011 & 11011 & 10001 & a_{3}+L \\
01100 & 00110 & 10110 & 11100 & a_{4}+L \\
01001 & 00011 & 10011 & 11001 & a_{5}+L \\
00101 & 01111 & \underline{11111} & 10101 & a_{6}+L \\
01101 & 00111 & 10111 & 11101 & a_{7}+L \\
\end{array} 
$$
11111在第7行第 3列 ,将 11111 译为第3列中最顶端的码字11010 ,同理 将10000 译为 10000.
\end{tcolorbox}


\newpage
 \begin{tcolorbox}[breakable,colback=blue!5!white,colframe=blue!75!black,
 title= 解答题]

设三元线性码 $ L $ 的生成矩阵为
$
G=\left(\begin{array}{llll}
1 & 1 & 1 & 0 \\
2 & 0 & 1 & 1
\end{array}\right) .
$

(1) 试求 $ L $ 的标准型的生成矩阵.

(2) 试求 $ L $ 的标准型的校验矩阵.

(3) 试利用伴随式译码方法对信道接收端接收到的字2121、1201、2222分别进行译码.
\tcblower
(1)
$
 G=\left(\begin{array}{llll}
1 & 1 & 1 & 0 \\
2 & 0 & 1 & 1
\end{array}\right) \rightarrow\left(\begin{array}{llll}
1 & 1 & 1 & 0 \\
0 & 1 & 2 & 1
\end{array}\right) \rightarrow\left(\begin{array}{llll}
1 & 0 & 2 & 2 \\
0 & 1 & 2 & 1
\end{array}\right)=\left(\begin{array}{ll|ll}
1 & 0 & 2 & 2 \\
0 & 1 & 2 & 1
\end{array}\right)=G^{\prime}
$

$ G^{\prime} $ 为 $ L $ 的标准型的生成矩阵.


(2) $ G^{\prime}=\left(\begin{array}{ll|ll}1 & 0 & 2 & 2 \\ 0 & 1 & 2 & 1\end{array}\right)=\left(I_{2} \mid A\right) $, 故 $ L $ 的标准型的校验矩阵为 $H= \left(-A^{T} \mid I_{2}\right)=\left(\begin{array}{llll}1 & 1 & 1 & 0 \\ 1 & 2 & 0 & 1\end{array}\right)$. 

(3)% $ L=\{0000,2011,1022,1110,0121,2102,2220,1201,0212\} $. 要对接收到的字 $ 2121 , 1201 $ , $2222$ 进行译码, 我们需要计算它们与 $ H $ 的乘积 $xH^{T} $, 以找到对应的综合校验向量.
%因为 $ d(L)=W(L)=3 $, 码 $ C $ 至多可以纠正一个错误, 所以 $ V(4,3) $ 中的 9 个重量不大于 1 的向量都是陪集头. 由于共有 $ \frac{3^{4}}{3^{2}}=9 $ 个陪集, 所以 $ V(4,3) $ 中的 9 个重量不大于 1 的向量恰好是所有陪集的代表元. 
码 $L$ 的伴随式列表为
\begin{center}
\begin{tabular}{c||c}
\hline 陪集头 & 伴随式 $xH^{T}$ \\
\hline 0000 & 00 \\
1000 & 11 \\
0100 & 12 \\
0010 & 10 \\
0001 & 01 \\
2000 & 22 \\
0200 & 21 \\
0020 & 20 \\
0002 & 02 \\
\hline
\end{tabular}
\end{center}

设信道接收端接收到的字2121、1201、2222分别为$x_1,x_2,x_3$,

$$
\begin{array}{ll}
x_{1} H^{T}=22, & a_{1}=2000, a_{1} H^{T}=22, x_{1}-a_{1}=0121 \\
x_{2} H^{T}=00, & a_{2}=0000, a_{2} H^{T}=00, x_{2}-a_{2}=1201 . \\
x_{3} H^{T}=02, & a_{3}=0002, a_{3} H^{T}=02, x_{3}-a_{3}=2220 .
\end{array}
$$

因此,$2121 $ 译码为 $ 2121-2000=0121$ . $1201 $ 译码为 $ 1201-0000=  1201$ ,$2222 $ 泽码为 $ 2222-0002=2220 $.
\end{tcolorbox}

\newpage

 \begin{tcolorbox}[breakable,colback=blue!5!white,colframe=blue!75!black,
 title= 解答题]

设二元线性码 $ L $ 的生成矩阵为
$
G=\left(\begin{array}{lllll}
1 & 0 & 0 & 1 & 1 \\
0 & 1 & 0 & 0 & 1 \\
0 & 0 & 1 & 0 & 1
\end{array}\right) .
$
试求 $ L $ 的重量分布.

\tcblower
$ G=\left(\begin{array}{lll|ll}
1 & 0 & 0 & 1 & 1 \\
0 & 1 & 0 & 0 & 1 \\
0 & 0 & 1 & 0 & 1
\end{array}\right)=\left(I_{3} \mid A\right) $, 故 $ L $ 的校验阵 $H= \left(-A^{T} \mid I_{3}\right)=\left(\begin{array}{lllll}1 & 0 & 0 & 1 & 0 \\ 1 & 1 & 1 & 0 & 1\end{array}\right)$. 根据定义知, $H$为线性码 $L$ 的对偶码 $L^{\perp}$的生成矩阵,于是 $ L^{\perp}=\{00000,10010,11101,01111\} $. 由于$L^{\perp}$ 是一个二元$[5,2]$线性码,则有
$$
W_{L^{\perp}}(z)=1+z^{2}+ z^{4}+z^{4}
$$

$L$ 是一个二元$[5,3]$线性码,则由二元线性码的 Mac Williams 恒等式知
$$
\begin{aligned}
W_{L}(z) & =\frac{1}{2^{2}}(1+z)^{5} W_{L^{\perp}}\left(\frac{1-z}{1+z}\right) \\
& =\frac{1}{4}(1+z)^{5}\left[1+\left(\frac{1-z}{1+z}\right)^{2}+2\left(\frac{1-z}{1+z}\right)^{4}\right] \\
& =1+3 z^{2}+3 z^{3}+z^{5} .
\end{aligned}
$$

因此线性码 $ L $ 的重量分布为
$$
A_{0}=1, A_{1}=0, A_{2}=3, A_{3}=3, A_{4}=0, A_{5}=1 .
$$
\end{tcolorbox}