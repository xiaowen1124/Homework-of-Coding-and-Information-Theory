\newpage
 \begin{tcolorbox}[breakable,colback=blue!5!white,colframe=blue!75!black,
 title= 填空题]

设二元 $ [4,2] $ 线性码 $ L=\{0000,1100,0011,1111\} $, 则 $ L $ 的对偶码为$\underline{\hspace{2em}}$.
\tcblower

一个线性码 $ L $ 的对偶码 $ L^{\perp} $ 定义为所有与 $ L $ 中所有码字正交的码字的集合.对于一个二元码,如果两个码字 $ x=\left(x_{1}, x_{2}, \ldots, x_{n}\right) $ 和 $ y=\left(y_{1}, y_{2}, \ldots, y_{n}\right) $ 的点积 $ x \cdot y= $ $ x_{1} y_{1}+x_{2} y_{2}+\ldots+x_{n} y_{n}=0 $ (模 2 计算),则这两个码字正交.

对于给定的线性码 $ L=\{0000,1100,0011,1111\} $ ,我们需要找到所有与 $ L $ 中每个码字都正交的码字构成的集合 $ L^{\perp} $ .  

$$
\left(\begin{array}{llll}
1 & 1 & 0&0 \\
0&0 & 1 & 1 \\
1 & 1 & 1&1
\end{array}\right) \rightarrow\left(\begin{array}{llll}
1 & 1 & 0&0 \\
0&0 & 1 & 1 \\
0 & 0 &0&0
\end{array}\right)
$$

$ L $ 的生成矩阵为 $ G=\left(\begin{array}{llll}1 & 1 & 0 & 0 \\ 0 & 0 & 1 & 1\end{array}\right) $.

 $ L^{\perp}=\left\{x G^{T}=0 \mid x \in V(n, q)\right\} $. 即 $ \left\{\begin{array}{l}x_{1}+x_{2}=0 \\ x_{3}+x_{4}=0\end{array}\right. $ 解得 $ \xi_{1}=(1,1,0,0), \xi_{2}=(0,0,1,1) $.$ \xi_{3}=(1,1,1,1), \xi_{4}=(0,0,0,0)$. $ L^{\perp}=L, L^{\perp} $ 也是 $ [4,2] $ 线性码.$ L^{\perp}=\{0000,1100,0011,1111\} $
 \end{tcolorbox}


 \begin{tcolorbox}[breakable,colback=blue!5!white,colframe=blue!75!black,
 title= 解答题]

设 $ E_{n} $ 是 $ V(n, 2) $ 中所有具有偶数重量的向量的集合. 证明: $ E_{n} $ 是线性码, 确定 $ E_{n} $ 的参数 $ [n, k, d] $ 以及其标准型的生成矩阵.
\tcblower
证明: $ \forall x, y \in E_{n}, d(x, y)=\omega(x-y), \omega(x-y)=\omega(x)+\omega(y)-2 \omega(x \cap y) $. 故 $ \omega(x+y) $ 为偶数, $ x+y=x-y, x+y \in E_{n}, E_{n} $ 是线性码. 由上一章课后题知 $ E_{n} $ 是一个 $ [n, n-1,2] $ 线性码, 标准生成阵为
$$
G=\left(\begin{array}{cccccc}
1 & 0 & 0 & \cdots & 0 & 1 \\
0 & 1 & 0 & \cdots & 0 & 1 \\
\vdots & \vdots & \vdots & & \vdots & \vdots \\
0 & 0 & 0 & \cdots & 1 & 1
\end{array}\right)
$$

设 $ L=\{x G \mid \forall x \in V(n-1, q)\} $, 对 $ n $ 用归纳法证明, 则 $ L $ 中的元均具有偶重量, 从而 $ L \subseteq E_{n} $.又 $ \operatorname{dim} L=\operatorname{dim} E_{n}=n-1 $, 故 $ L=E_{n} $.
$ \left(\forall x \in L, x=v_{j 1}+v_{j 2}+\cdots+v_{j k}\right. $, 系数取在 $ F_{2} $ 中 $ \{0,1\}, k=2 $ 时, $ x=v_{j 1}+v_{j 2}, \omega(x)= $ $ \omega\left(v_{j 1}\right)+\omega\left(v_{j 2}\right)-2 \omega\left(v_{j 1} \cap v_{j 2}\right) $ 为偶数容易用归纳法证得)
 \end{tcolorbox}

\newpage
 \begin{tcolorbox}[breakable,colback=blue!5!white,colframe=blue!75!black,
 title= 解答题]

设 $ E_{n} $ 是 $ V(n, 2) $ 中所有具有偶数重量的向量的集合. 证明: $ E_{n} $ 是线性码, 确定 $ E_{n} $ 的参数 $ [n, k, d] $ 以及其标准型的生成矩阵.
\tcblower
 (1)要证明 $ E_{n} $ 是线性码,我们需要证明对于任意两个向量 $ x, y \in E_{n} $ ,它们的加法 $ x+y $ (在 $ \mathbb{F}_{2} $上是按位异或) 仍然在 $ E_{n} $ 中.对于任何二元向量,其重量的公式 $ \omega(x+y) $ 可以通过 $ \omega(x)+\omega(y)-2 \omega(x \cap $ $ y) $ 来计算,其中 $ \omega(x \cap y) $ 是 $ x $ 和 $ y $ 同时为 1 的位置数.因 $ x, y $ 的重量都是偶数, $ \omega(x)+\omega(y) $ 也是偶数.由于 $ \omega(x \cap y) $ 是整数, $ 2 \omega(x \cap y) $ 一定是偶数,从而 $ \omega(x+y) $ 是偶数,证明 $ x+y $ 也在 $ E_{n} $ 中.得证.

(2)码的长度 $ n $:码的长度 $ n $ 指的是每个码字的位数,由于 $ E_{n} $ 包括 $ V(n, 2) $ 中所有具有偶数重量的向量,每个向量自然是长度为 $ n $ 的向量.

维数 $ k $ 表示线性码的生成矩阵的行数,也即是该码作为向量空间的基的向量数目.在 $ E_{n} $ 的情况中,生成矩阵 $ G $ 可以构造为长度为 $ n $ 且每行保证向量总重量为偶数的矩阵,具体形式如下

$$
G=\left(\begin{array}{cccccc}
1 & 0 & 0 & \cdots & 0 & 1 \\
0 & 1 & 0 & \cdots & 0 & 1 \\
\vdots & \vdots & \vdots & & \vdots & \vdots \\
0 & 0 & 0 & \cdots & 1 & 1
\end{array}\right)
$$

这个矩阵的每一行代表一个基向量,其中最后一个元素是其它所有元素的和(确保总重量为偶数).

考虑到 $ E_{n} $ 是所有偶数重量向量的集合,若 $ n $ 是奇数,则所有 $ n $ 位向量中重量为奇数的向量不能直接包含在 $ E_{n} $ 中,但可以通过其它向量的线性组合得到(例如全“ 1 "向量可以通过其它所有位为1且总数为奇数的向量异或得到).

因此, $ E_{n} $ 实际能够生成的独立向量数为 $ n-1 $ ,即除去一个线性相关的向量(例如,全“1”向量),余下的向量能够生成所有偶数重量的向量.这意味着 $ E_{n} $ 作为子空间的维数是 $ n-1 $ .


最小汉明距离 $ d $ 是指码中任意两个不同码字间至少有 $ d $ 个位是不同的.对于 $ E_{n} $ ,因为所有码字的重量都是偶数,所以任何两个不同的码字至少要在两个位置上有差异,以确保它们的总重量变化保持为偶数(如果仅一个位不同,一个码字的重量将由偶数变为奇数或反之,不满足偶数重量的要求) .因此, $ E_{n} $ 的最小汉明距离至少是 2 .

综上所述, $ E_{n} $ 是一个 $ [n, n-1,2] $ 线性码.

(3)标准型的生成矩阵 $ G $ 的具体形式是:
$
G=\left(\begin{array}{cccccc}
1 & 0 & 0 & \cdots & 0 & 1 \\
0 & 1 & 0 & \cdots & 0 & 1 \\
0 & 0 & 1 & \cdots & 0 & 1 \\
\vdots & \vdots & \vdots & \ddots & \vdots & \vdots \\
0 & 0 & 0 & \cdots & 1 & 1
\end{array}\right)
$

这里, $ G $ 是一个 $ (n-1) \times n $ 矩阵,其中前 $ n-1 $ 列是 $ I_{n-1}(n-1 $ 阶单位矩阵),最后一列是全例 (如果 $ n $ 是奇数) 或全 0 列 (如果 $ n $ 是偶数),以保持每行的重量为偶数.
 
 \end{tcolorbox}





  \begin{tcolorbox}[breakable,colback=blue!5!white,colframe=blue!75!black,
 title= 解答题]

 证明: 对于任意一个二元线性码 $ L $, 一定满足下列条件之一.
 
(1) $L$ 中所有码字都具有偶数重量;

(2) $ L $ 中一半码字具有偶数重量, 另一半码字具有奇数重量.
\tcblower


证明: 设 $ L $ 是一个二元 $ [n, k] $ 线性码, 且 $ L $ 中存在码字 $ x_{0} $, 使得 $ \omega\left(x_{0}\right) $ 为奇数. 令 $ L_{1}= $ $ \{x \in L \mid \omega(x) $ 是偶数 $ \}, L_{2}=\{x \in L \mid \omega(x) $ 是奇数 $ \}, x_{0}+L_{1}=\left\{x_{0}+x \mid \forall x \in L_{1}\right\}, x_{0}+L_{1} $ 中元素重量为奇数, $ x_{0}+L_{1} \subseteq L_{2} $ 同理 $ x_{0}+L_{2} \subseteq L_{1} $.
$
\left|L_{2}\right| \geq\left|x_{0}+L_{1}\right|=\left|L_{1}\right| \geq\left|x_{0}+L_{2}\right|=\left|L_{2}\right| \Rightarrow\left|L_{1}\right|=\left|L_{2}\right| .
$
(否定一个证另一个)
 \end{tcolorbox}

   \begin{tcolorbox}[breakable,colback=blue!5!white,colframe=blue!75!black,
 title= 解答题]

 证明: 对于任意一个二元线性码 $ L $, 一定满足下列条件之一.
 
(1) $L$ 中所有码字都具有偶数重量;

(2) $ L $ 中一半码字具有偶数重量, 另一半码字具有奇数重量.
\tcblower

给定条件是 $ L $ 是一个二元 $ [n, k] $ 线性码.这意味着 $ L $ 包含 $ 2^{k} $ 个码字,每个码字长度为 $ n $ .

定义两个集合: $ L_{1}=\{x \in L \mid \omega(x) $ 是偶数 $ \} $, $ L_{2}=\{x \in L \mid \omega(x) $ 是奇数 $ \} $.

假设在 $ L $ 中存在至少一个码字 $ x_{0} $ 使得 $ \omega\left(x_{0}\right) $ 是奇数.我们将使用这个码字来构建集合映射.

对于 $ L_{1} $ 中的任意码字 $ x $ ,由于 $ x_{0} $ 是奇数重量, $ x $ 是偶数重量,那么 $ x_{0}+x $ 将是奇数重量(偶数与奇数相加结果为奇数).因此, $ x_{0}+L_{1}=\left\{x_{0}+x \mid x \in L_{1}\right\} \subseteq L_{2} $ .

同样,对于 $ L_{2} $ 中的任意码字 $ y, x_{0}+y $ 将是偶数重量 (奇数与奇数相加结果为偶数) .因此, $ x_{0}+L_{2}=\left\{x_{0}+y \mid y \in L_{2}\right\} \subseteq L_{1} $ .


由于 $ x_{0}+L_{1} \subseteq L_{2} $ 且 $ x_{0}+L_{2} \subseteq L_{1} $ ,我们知道 $ x_{0}+L_{1} $ 和 $ x_{0}+L_{2} $ 分别是 $ L_{2} $ 和 $ L_{1} $ 的一部分,且由于码的线性属性,加法 $ x_{0}+x $ (其中 $ x $ 是 $ L $ 的任意成员) 是双射的(即一一对应且可逆).因此, $ \left|x_{0}+L_{1}\right|=\left|L_{1}\right| $ 和 $ \left|x_{0}+L_{2}\right|=\left|L_{2}\right| $ .

由 $ \left|L_{2}\right| \geq\left|x_{0}+L_{1}\right|=\left|L_{1}\right| $ 和 $ \left|L_{1}\right| \geq\left|x_{0}+L_{2}\right|=\left|L_{2}\right| $ ,我们得出 $ \left|L_{1}\right|=\left|L_{2}\right| $ .

这表明如果 $ L $ 中存在至少一个奇数重量的码字,那么偶数重量的码字和奇数重量的码字数量必然相等,即 $ \left|L_{1}\right|=\left|L_{2}\right| $ .如果 $ L $ 中所有码字都具有偶数重量,那么 $ L_{2} $ 为空集,从而也满足题目中的条件之一.

综上所述,对于任意一个二元线性码 $ L $ ,要么所有码字都具有偶数重量,要么一半码字具有偶数重量,另一半码字具有奇数重量.

 \end{tcolorbox}