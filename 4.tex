
\section{第四次课后作业}

\begin{tcolorbox}[breakable,colback=blue!5!white,colframe=blue!75!black,
 title= 单选题]
 码 $ C $ 是前缀码是码 $ C $ 是即时码的( ).

A. 充分条件

B. 充分必要条件

C. 必要条件

\tcblower
(B)

$ \Leftarrow $ : 假设 $ C $ 是一个码元集, 若 $ \mathrm{C} $ 不是前缀码, 则存在码字 $ c_{i}, C_{j} $,使得 $ c_{i} $ 是 $ c_{j} $ 的前缀, 在一个含有 $ c_{i} $ 的码字串中, 从左到右, 当 $ c_{i} $ 出现时,只有当 $ c_{i} $ 后面出现部分, 连同 $ c_{i} $ 不是 $ c_{j} $ 时才能把 $ c_{i} $ 还原;若 $ c_{i} $以及连同后面部分是 $ c_{j} $ 时, 不能把 $ c_{i} $ 还原, 应该把 $ c_{j} $ 还原, 因此 $ C $ 不是即时码,矛盾.故即时码一定为前缀码.
 
$ \Rightarrow $ : 若 $ C $ 不是即时码, 则从左到右, 出现一个码字 $ c_{i} $, 还原为消息字母时, 依赖于后面的字符串, 即存在另一个码字 $ c_{j} $, 使得 $ c_{i} $ 是 $ c_{j} $ 的前缀, 从而C不是前缀码.

\end{tcolorbox}


\begin{tcolorbox}[breakable,colback=blue!5!white,colframe=blue!75!black,
 title= 单选题]
码字长度为 $ \left\{\ell_{1}, \ell_{2}, \cdots, \ell_{a}\right\} $ 的码为即时码是 $ \left\{\ell_{1}, \ell_{2}, \cdots, \ell_{a}\right\} $ 满足Kraft不等式的( ).

A. 充分条件

B. 充分必要条件

C. 必要条件

\tcblower

 若 $ \ell_{1}, \ell_{2}, \cdots, \ell_{2} $ 满足 Kraft不等式, 则必存在码字长度为 $ \ell_{1}, \ell_{2}, \cdots, \ell_{a} $ 的即时码. 如果一个码的码字长度满足 Kraft 不等式, 但它不一定是即时码.

如: 考虑二元码 $ C=\{0,11,100,110\} $,码字长度分别为 $ 1,2,3,3 $,因为$|\mathscr{U}|=2$, 我们有
$$
\frac{1}{2}+\frac{1}{2^{2}}+\frac{1}{2^{3}}+\frac{1}{2^{3}}=1,
$$
所以, 它的码字长度满足kraft 不等式.
但这个码并不是即时的(不是前缀码), 因为码字11是码字110的前缀.
但根据 $ 1,2,3 $, 3 可构造一个即时码,如 $ \{0,10,110,111\} $或 $ \{1,01,001,000\} $.

因此码字长度为 $ \left\{\ell_{1}, \ell_{2}, \cdots, \ell_{a}\right\} $ 的码为即时码是 $ \left\{\ell_{1}, \ell_{2}, \cdots, \ell_{a}\right\} $ 满足Kraft不等式的充分条件.(A)

\end{tcolorbox}



\begin{tcolorbox}[breakable,colback=blue!5!white,colframe=blue!75!black,
 title= 解答题]
 
下面的码是否是即时码? 是否是唯一可译码?\\
(1) $ C=\{0,10,1100,1101,1110,1111\} $.\\
(2) $ C=\{0,10,110,1110,1011,1101\} $.

\tcblower

(1) 在这个码集中,没有任何码字是另一个码字的前缀.因此,每个码字的开始都唯一标识了一个码字,没有歧义,因此是即时码.由于是即时码,它自然也是唯一可译码.即时码的属性保证了解码过程中的唯一性. 故 C 是前缀码, 是即时码, 是唯一可译码.
 
(2) C不是前缀码, 因为码字10是码字1011的前缀, 故C不是即时码.C不是唯一可译码.因为根据如下
字符串得知:一个给定的编码序列中可能会解码出两种不同的消息,表明在这个特定的码集不是唯一可译码.
$$
\frac{0}{a} \frac{10}{b} \frac{110}{c} \frac{1110}{d} \frac{1011}{e} \frac{1101}{\mathrm{f}} $$
$$
\frac{0}{a} \frac{1011}{e} \frac{0}{a} \frac{1110}{d} \frac{1011}{e} \frac{1101}{\mathrm{f}}
$$

\end{tcolorbox}


\begin{tcolorbox}[breakable,colback=blue!5!white,colframe=blue!75!black,
 title= 解答题]
 
判断是否存在即时码具有以下的基数和码字长度, 如果有, 试构造出一个这样的码.\\
(1) $ r=2 $, 长度: 1,3,3,3,4,4.\\
(2) $ r=3 $ ,长度: 1,1,2,2,3,3,3.\\
(3) $ r=5 $, 长度: $ 1,1,1,1,1,8,9 $.


\tcblower

(1) $ r=2 $, 长度: $ 1,3,3,3,4,4 $.

首先,我们计算Kraft和:
$$
\sum_{i=1}^{6} 2^{-\ell_{i}}=2^{-1}+3 \times 2^{-3}+2 \times 2^{-4}=1
$$
Kraft和等于1,满足Kraft不等式,因此存在即时码. 下面构造一个即时码:\\
长度1的码字:0\\
 长度3的码字: $ 100,101,110 $\\
 长度4的码字: 1110,1111

 $$
\begin{array}{lllll}
u_{1,1}=0 & 0 & & &\\
u_{3,1,1}=1 & u_{3,1,2}=0 & u_{3,1,3}=0 & (1,0,0) &\\
u_{3,2,1}=1 & u_{3,2,2}=0 & u_{3,2,3}=1 & (1,0,1) &\\
u_{3,3,1}=1 & u_{3,3,2}=1 & u_{3,3,3}=0 & (1,1,0) &\\
u_{4,1,1}=1 & u_{4,1,2}=1 & u_{4,1,3}=1 & u_{4,1,4}=0 & (1,1,1,0)\\
u_{4,2,1}=1 & u_{4,2,2}=1 & u_{4,2,3}=1 & u_{4,2,4}=1& (1,1,1,1)
\end{array}
$$
故此即时码为
$$
\{0,100,101,110,1110,1111)\}
$$

(2) $ r=3 $, 长度: 1, 1, 2, 2, 3, 3, 3 .

接下来,我们计算Kraft和:
$$
\sum_{i=1}^{7} 3^{-\ell_{i}}=2 \times 3^{-1}+2 \times 3^{-2}+3 \times 3^{-3}=\frac{2}{3}+\frac{2}{9}+\frac{3}{27}=\frac{2}{3}+\frac{2}{9}+\frac{1}{9}=1
$$
Kraft和等于1,满足Kraft不等式,因此存在即时码。下面构造一个即时码:\\
长度1的码字:0,1\\
长度2的码字: 20,21\\
长度3的码字: $ 220,221,222 $

$$
\begin{array}{llrl}
u_{1,1}=0 & u_{1,2}=1 & 0,1&\\
u_{2,1,1}=2 & u_{2,1,2}=0 & (2,0) & \\
u_{2,2,1}=2 & u_{2,2,2}=1 & (2,1) & \\
u_{3,1,1}=2 & u_{3,1,2}=2 & u_{3,1,3}=0 & (2,2,0) \\
u_{3,2,1}=2 & u_{3,2,2}=2 & u_{3,2,3}=1 & (2,2,1) \\
u_{3,3,1}=2 & u_{3,3,2}=2 & u_{3,3,3}=2 & (2,2,2)
\end{array}
$$

故此即时码为
$$
\{0,1,20,21,220,221,222\}
$$

(3)我们计算Kraft和:
$$
\sum_{i=1}^{7} 5^{-\ell_{i}} =5 \times \frac{1}{5}+\frac{1}{5^{8}}+\frac{1}{5^{9}}>1
$$
故这样的即时码不存在.


\end{tcolorbox}



